\section{Diskussion}
\label{sec:Diskussion}


Das Ergebnis für den Elastizitätsmodul des runden
Stabes bei einseitiger Einspannung ($E=(85,4 \pm 1,1) \cdot 10^{9} \,\si{\kilo\gram\per\meter\per\second\squared}$)
lässt vermuten, dass es sich bei dem Material des
Stabes um Messing handelt. Der Literaturwert 
des Elastizitätsmoduls bei Messing beträgt
$(78-123)\cdot 10^{9} \,\si{\kilo\gram\per\meter\per\second\squared}$(Siehe ZITAT).
Die Ergebnisse bei der beidseitigen Einspannung weichen
jedoch um (PROZENT) von der unteren Grenze des Literaturwertes
auf der rechten Seite ab.


Das Ergebnis für den Elastizitätsmodul des eckigen
Stabes bei einseitiger Einspannung ($E=(76,9 \pm 1,9) \cdot 10^{9} \,\si{\kilo\gram\per\meter\per\second\squared}$)
lässt vermuten, dass es sich bei dem Material des
Stabes um Aluminium handelt. Der Literaturwert 
des Elastizitätsmoduls bei Aluminium beträgt
$(70)\cdot 10^{9} \,\si{\kilo\gram\per\meter\per\second\squared}$ (Siehe (ZITAT)).
Das aus der Messung bestimmte Elastizitätsmodul weicht also um (PROZENT) ab.
Bei der beidseitigen Messung treten Abweichungen von (PROZENT)
auf der linken und (PROZENT) auf der rechten Seite auf.


Die Abweichungen lassen sich durch statischtische Fehler
und systematische Fehler bei der Messung erklären. 
Wahrscheinlich ist, dass Ungenauigkeiten beim Ablesen der Messuhr
und beim kalibrieren dieser aufgetreten sind.
Der zweite Stab kann durch die großen Abweichungen
für die einseitige und beidseitige Einspannung nicht sicher
als Aluminium identifiziert werden.